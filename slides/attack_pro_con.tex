%!TEX root = ../presentation.tex

%Fordele/ulemper ved attack trees
%Subtræer der giver adgang til noget forskelligt (SM, ECM, physical)

\begin{frame}\frametitle{title}
  % Attack trees will describe the cause-effect relation between parts of an attack:
  % I must do a before I do B.
  % This describes the goal of the attack as well as any prerequisites for performing that specific attack.
  % Problematic to represent the correlation between a cause and effect when intertwined
\end{frame}

%Genbrug af subtræer, referencer (?)

\begin{frame}\frametitle{title}
  % To solve the above (relation between cause and effect) subtrees could have been used.
  % Subtrees could represent different groups of nodes, such as ways to gain access to the system or exploitations.
  % The larger trees could then have been represented using combinations of these trees.
  % This might have simplified the visual appearance of the large trees.
\end{frame}

%Størrelse -- bliver meget hurtigt store
%delebørn -- multiple parents

\begin{frame}\frametitle{title}
  % Besides representing the order between a nodes elements (A before B), it might also be nice to show how one element is related to multiple elements.
  % In the current representation you are required to look for these relations, instead of having them appear visually.
  % 
\end{frame}
